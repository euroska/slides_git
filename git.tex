\documentclass{beamer}

\usepackage[utf8]{inputenc}
\usepackage{default}
\usepackage{graphicx}
\usepackage{listings}
\usepackage{tikz}
% \usepackage[crop=off]{auto-pst-pdf}

\title{Advenced GIT}
\author{Martin Miksanik}
\subject{Advanced git}
\date{\today}

\usetheme{Bergen}
\usecolortheme{beetle}

\begin{document}

\begin{frame}
  \titlepage
  \vspace{1cm}
  Mozno zaroven testovat na: $http://gitlab.mmik.cz$
\end{frame}


\frame{\tableofcontents}

\section{Úvod}
\begin{frame}
  \frametitle{Úvod}
  \begin{itemize}
    \item Proc se to ucit, kdyz to jde naklikat?
    \item Obří množství funkcí/parametru
    \item Široké možnosti konfigurace - diff, whitespace atd.
    \item Jednoduše nastavitelné hooky
    \item Snadna CI
    
    \item https://git-scm.com/
    \item git help    
  \end{itemize}
\end{frame}

\section{Základní operace}
\begin{frame}
  \frametitle{Základní operace}
  \begin{itemize}
    \item Clone / Fetch
    \item Add / rm
    \item Commit / Reset
    \item Push / Pull
    \item Checkout
    \item Branch 
    \item Merge / Rebase
    \item Revert
    \item Stash
    \item ...
  \end{itemize}
  
  \begin{itemize}
   \item Bare git repozitář
   \item Repository s index a WORK\_TREE
  \end{itemize}
\end{frame}

\subsection{Clone / Fetch}
\begin{frame}
  \frametitle{Clone / Fetch}
  \begin{block}{Clone}
    git clone https://github.com/euroska/slides\_git.git
    \begin{itemize}
     \item Stáhne vzdálené repository (fetch)
     \item Nastaví remote origin
     \item Nastavi index (branche, HEAD)
    \end{itemize}
  \end{block}  
  \begin{block}{Fetch}
    git fetch origin -a
    \begin{itemize}
      \item Stahne vsechny zmeny ze vzdalenych adresaru, ale neaplikuje je
      \item Vzdalene repozitare je mozno vyuzivat (git rebase origin/master)
    \end{itemize}   
  \end{block}
\end{frame}

\subsection{Prace se stromem}
\begin{frame}
  \frametitle{Strom}
  \begin{figure}
    % Graphic for TeX using PGF
% Title: /home/sauron/Projects/slides/git/structure.dia
% Creator: Dia v0.97.3
% CreationDate: Tue Nov  8 00:17:00 2016
% For: sauron
% \usepackage{tikz}
% The following commands are not supported in PSTricks at present
% We define them conditionally, so when they are implemented,
% this pgf file will use them.
\ifx\du\undefined
  \newlength{\du}
\fi
\setlength{\du}{15\unitlength}
\begin{tikzpicture}
\pgftransformxscale{1.000000}
\pgftransformyscale{-1.000000}
\definecolor{dialinecolor}{rgb}{0.000000, 0.000000, 0.000000}
\pgfsetstrokecolor{dialinecolor}
\definecolor{dialinecolor}{rgb}{1.000000, 1.000000, 1.000000}
\pgfsetfillcolor{dialinecolor}
\pgfsetlinewidth{0.100000\du}
\pgfsetdash{}{0pt}
\pgfsetdash{}{0pt}
\pgfsetmiterjoin
\pgfsetbuttcap
{
\definecolor{dialinecolor}{rgb}{0.000000, 0.000000, 0.000000}
\pgfsetfillcolor{dialinecolor}
% was here!!!
\pgfsetarrowsend{stealth}
\definecolor{dialinecolor}{rgb}{0.000000, 0.000000, 0.000000}
\pgfsetstrokecolor{dialinecolor}
\pgfpathmoveto{\pgfpoint{20.759766\du}{13.019531\du}}
\pgfpathcurveto{\pgfpoint{20.259766\du}{12.019531\du}}{\pgfpoint{20.037180\du}{12.000000\du}}{\pgfpoint{18.537180\du}{12.000000\du}}
\pgfusepath{stroke}
}
\pgfsetlinewidth{0.300000\du}
\pgfsetdash{}{0pt}
\pgfsetdash{}{0pt}
\pgfsetmiterjoin
\pgfsetbuttcap
{
\definecolor{dialinecolor}{rgb}{0.000000, 0.000000, 0.000000}
\pgfsetfillcolor{dialinecolor}
% was here!!!
\pgfsetarrowsend{stealth}
\definecolor{dialinecolor}{rgb}{0.000000, 0.000000, 0.000000}
\pgfsetstrokecolor{dialinecolor}
\pgfpathmoveto{\pgfpoint{21.000000\du}{12.951172\du}}
\pgfpathcurveto{\pgfpoint{21.000000\du}{12.451172\du}}{\pgfpoint{21.000000\du}{10.548828\du}}{\pgfpoint{21.000000\du}{10.048828\du}}
\pgfusepath{stroke}
}
\pgfsetlinewidth{0.300000\du}
\pgfsetdash{}{0pt}
\pgfsetdash{}{0pt}
\pgfsetmiterjoin
\pgfsetbuttcap
{
\definecolor{dialinecolor}{rgb}{0.000000, 0.000000, 0.000000}
\pgfsetfillcolor{dialinecolor}
% was here!!!
\pgfsetarrowsend{stealth}
\definecolor{dialinecolor}{rgb}{0.000000, 0.000000, 0.000000}
\pgfsetstrokecolor{dialinecolor}
\pgfpathmoveto{\pgfpoint{21.000000\du}{8.950345\du}}
\pgfpathcurveto{\pgfpoint{21.000000\du}{7.950345\du}}{\pgfpoint{20.500000\du}{7.549655\du}}{\pgfpoint{20.500000\du}{6.049655\du}}
\pgfusepath{stroke}
}
\pgfsetlinewidth{0.100000\du}
\pgfsetdash{}{0pt}
\pgfsetdash{}{0pt}
\pgfsetmiterjoin
\pgfsetbuttcap
{
\definecolor{dialinecolor}{rgb}{0.000000, 0.000000, 0.000000}
\pgfsetfillcolor{dialinecolor}
% was here!!!
\pgfsetarrowsend{stealth}
\definecolor{dialinecolor}{rgb}{0.000000, 0.000000, 0.000000}
\pgfsetstrokecolor{dialinecolor}
\pgfpathmoveto{\pgfpoint{17.612874\du}{11.612874\du}}
\pgfpathcurveto{\pgfpoint{16.612874\du}{10.612874\du}}{\pgfpoint{18.544471\du}{6.803686\du}}{\pgfpoint{20.044471\du}{5.803686\du}}
\pgfusepath{stroke}
}
\pgfsetlinewidth{0.100000\du}
\pgfsetdash{}{0pt}
\pgfsetdash{}{0pt}
\pgfsetmiterjoin
\pgfsetbuttcap
{
\definecolor{dialinecolor}{rgb}{0.000000, 0.000000, 0.000000}
\pgfsetfillcolor{dialinecolor}
% was here!!!
\pgfsetarrowsend{stealth}
\definecolor{dialinecolor}{rgb}{0.000000, 0.000000, 0.000000}
\pgfsetstrokecolor{dialinecolor}
\pgfpathmoveto{\pgfpoint{21.173339\du}{12.979983\du}}
\pgfpathcurveto{\pgfpoint{21.673339\du}{11.479983\du}}{\pgfpoint{22.826661\du}{12.020017\du}}{\pgfpoint{23.326661\du}{10.520017\du}}
\pgfusepath{stroke}
}
\pgfsetlinewidth{0.300000\du}
\pgfsetdash{}{0pt}
\pgfsetdash{}{0pt}
\pgfsetmiterjoin
\pgfsetbuttcap
{
\definecolor{dialinecolor}{rgb}{0.000000, 0.000000, 0.000000}
\pgfsetfillcolor{dialinecolor}
% was here!!!
\pgfsetarrowsend{stealth}
\definecolor{dialinecolor}{rgb}{0.000000, 0.000000, 0.000000}
\pgfsetstrokecolor{dialinecolor}
\pgfpathmoveto{\pgfpoint{20.500000\du}{4.950134\du}}
\pgfpathcurveto{\pgfpoint{20.500000\du}{3.950134\du}}{\pgfpoint{19.866638\du}{3.533448\du}}{\pgfpoint{20.366638\du}{1.533448\du}}
\pgfusepath{stroke}
}
\pgfsetlinewidth{0.100000\du}
\pgfsetdash{}{0pt}
\pgfsetdash{}{0pt}
\pgfsetmiterjoin
\pgfsetbuttcap
{
\definecolor{dialinecolor}{rgb}{0.000000, 0.000000, 0.000000}
\pgfsetfillcolor{dialinecolor}
% was here!!!
\pgfsetarrowsend{stealth}
\definecolor{dialinecolor}{rgb}{0.000000, 0.000000, 0.000000}
\pgfsetstrokecolor{dialinecolor}
\pgfpathmoveto{\pgfpoint{19.949990\du}{5.500000\du}}
\pgfpathcurveto{\pgfpoint{19.285990\du}{5.500000\du}}{\pgfpoint{19.245972\du}{4.991944\du}}{\pgfpoint{18.745972\du}{3.991944\du}}
\pgfusepath{stroke}
}
\pgfsetlinewidth{0.100000\du}
\pgfsetdash{}{0pt}
\pgfsetdash{}{0pt}
\pgfsetmiterjoin
\pgfsetbuttcap
{
\definecolor{dialinecolor}{rgb}{0.000000, 0.000000, 0.000000}
\pgfsetfillcolor{dialinecolor}
% was here!!!
\pgfsetarrowsend{stealth}
\definecolor{dialinecolor}{rgb}{0.000000, 0.000000, 0.000000}
\pgfsetstrokecolor{dialinecolor}
\pgfpathmoveto{\pgfpoint{18.008056\du}{3.745972\du}}
\pgfpathcurveto{\pgfpoint{16.008056\du}{4.745972\du}}{\pgfpoint{15.754028\du}{2.991944\du}}{\pgfpoint{16.254028\du}{1.991944\du}}
\pgfusepath{stroke}
}
\pgfsetlinewidth{0.100000\du}
\pgfsetdash{}{0pt}
\pgfsetdash{}{0pt}
\pgfsetmiterjoin
\pgfsetbuttcap
{
\definecolor{dialinecolor}{rgb}{0.000000, 0.000000, 0.000000}
\pgfsetfillcolor{dialinecolor}
% was here!!!
\pgfsetarrowsend{stealth}
\definecolor{dialinecolor}{rgb}{0.000000, 0.000000, 0.000000}
\pgfsetstrokecolor{dialinecolor}
\pgfpathmoveto{\pgfpoint{17.049655\du}{1.500000\du}}
\pgfpathcurveto{\pgfpoint{18.377655\du}{1.500000\du}}{\pgfpoint{18.622345\du}{1.000000\du}}{\pgfpoint{19.950345\du}{1.000000\du}}
\pgfusepath{stroke}
}
\pgfsetlinewidth{0.100000\du}
\pgfsetdash{}{0pt}
\pgfsetdash{}{0pt}
\pgfsetmiterjoin
\pgfsetbuttcap
{
\definecolor{dialinecolor}{rgb}{0.000000, 0.000000, 0.000000}
\pgfsetfillcolor{dialinecolor}
% was here!!!
\pgfsetarrowsend{stealth}
\definecolor{dialinecolor}{rgb}{0.000000, 0.000000, 0.000000}
\pgfsetstrokecolor{dialinecolor}
\pgfpathmoveto{\pgfpoint{23.500000\du}{9.450345\du}}
\pgfpathcurveto{\pgfpoint{23.500000\du}{8.950345\du}}{\pgfpoint{23.633311\du}{8.533243\du}}{\pgfpoint{23.133311\du}{6.533243\du}}
\pgfusepath{stroke}
}
\pgfsetlinewidth{0.100000\du}
\pgfsetdash{}{0pt}
\pgfsetdash{}{0pt}
\pgfsetmiterjoin
\pgfsetbuttcap
{
\definecolor{dialinecolor}{rgb}{0.000000, 0.000000, 0.000000}
\pgfsetfillcolor{dialinecolor}
% was here!!!
\pgfsetarrowsend{stealth}
\definecolor{dialinecolor}{rgb}{0.000000, 0.000000, 0.000000}
\pgfsetstrokecolor{dialinecolor}
\pgfpathmoveto{\pgfpoint{22.449782\du}{6.000000\du}}
\pgfpathcurveto{\pgfpoint{22.117782\du}{6.000000\du}}{\pgfpoint{21.366553\du}{6.033790\du}}{\pgfpoint{21.866553\du}{4.033790\du}}
\pgfusepath{stroke}
}
\pgfsetlinewidth{0.100000\du}
\pgfsetdash{}{0pt}
\pgfsetdash{}{0pt}
\pgfsetmiterjoin
\pgfsetbuttcap
{
\definecolor{dialinecolor}{rgb}{0.000000, 0.000000, 0.000000}
\pgfsetfillcolor{dialinecolor}
% was here!!!
\pgfsetarrowsend{stealth}
\definecolor{dialinecolor}{rgb}{0.000000, 0.000000, 0.000000}
\pgfsetstrokecolor{dialinecolor}
\pgfpathmoveto{\pgfpoint{23.525256\du}{6.000000\du}}
\pgfpathcurveto{\pgfpoint{24.023256\du}{6.000000\du}}{\pgfpoint{25.500000\du}{6.025256\du}}{\pgfpoint{25.500000\du}{4.525256\du}}
\pgfusepath{stroke}
}
\pgfsetlinewidth{0.100000\du}
\pgfsetdash{}{0pt}
\pgfsetdash{}{0pt}
\pgfsetmiterjoin
\pgfsetbuttcap
{
\definecolor{dialinecolor}{rgb}{0.000000, 0.000000, 0.000000}
\pgfsetfillcolor{dialinecolor}
% was here!!!
\pgfsetarrowsend{stealth}
\definecolor{dialinecolor}{rgb}{0.000000, 0.000000, 0.000000}
\pgfsetstrokecolor{dialinecolor}
\pgfpathmoveto{\pgfpoint{22.000000\du}{2.949789\du}}
\pgfpathcurveto{\pgfpoint{22.000000\du}{1.949789\du}}{\pgfpoint{21.889058\du}{2.389058\du}}{\pgfpoint{20.889058\du}{1.389058\du}}
\pgfusepath{stroke}
}
\pgfsetlinewidth{0.100000\du}
\pgfsetdash{}{0pt}
\pgfsetdash{}{0pt}
\pgfsetmiterjoin
\pgfsetbuttcap
{
\definecolor{dialinecolor}{rgb}{0.000000, 0.000000, 0.000000}
\pgfsetfillcolor{dialinecolor}
% was here!!!
\pgfsetarrowsend{stealth}
\definecolor{dialinecolor}{rgb}{0.000000, 0.000000, 0.000000}
\pgfsetstrokecolor{dialinecolor}
\pgfpathmoveto{\pgfpoint{25.500000\du}{3.450162\du}}
\pgfpathcurveto{\pgfpoint{25.500000\du}{1.950162\du}}{\pgfpoint{23.888794\du}{1.611206\du}}{\pgfpoint{22.388794\du}{3.111206\du}}
\pgfusepath{stroke}
}
\definecolor{dialinecolor}{rgb}{1.000000, 1.000000, 1.000000}
\pgfsetfillcolor{dialinecolor}
\pgfpathellipse{\pgfpoint{18.000000\du}{12.000000\du}}{\pgfpoint{0.500000\du}{0\du}}{\pgfpoint{0\du}{0.500000\du}}
\pgfusepath{fill}
\pgfsetlinewidth{0.100000\du}
\pgfsetdash{}{0pt}
\pgfsetdash{}{0pt}
\definecolor{dialinecolor}{rgb}{0.000000, 0.000000, 0.000000}
\pgfsetstrokecolor{dialinecolor}
\pgfpathellipse{\pgfpoint{18.000000\du}{12.000000\du}}{\pgfpoint{0.500000\du}{0\du}}{\pgfpoint{0\du}{0.500000\du}}
\pgfusepath{stroke}
% setfont left to latex
\definecolor{dialinecolor}{rgb}{0.000000, 0.000000, 0.000000}
\pgfsetstrokecolor{dialinecolor}
\node at (18.000000\du,12.222500\du){B1};
\definecolor{dialinecolor}{rgb}{1.000000, 1.000000, 1.000000}
\pgfsetfillcolor{dialinecolor}
\pgfpathellipse{\pgfpoint{21.000000\du}{13.500000\du}}{\pgfpoint{0.500000\du}{0\du}}{\pgfpoint{0\du}{0.500000\du}}
\pgfusepath{fill}
\pgfsetlinewidth{0.100000\du}
\pgfsetdash{}{0pt}
\pgfsetdash{}{0pt}
\definecolor{dialinecolor}{rgb}{0.000000, 0.000000, 0.000000}
\pgfsetstrokecolor{dialinecolor}
\pgfpathellipse{\pgfpoint{21.000000\du}{13.500000\du}}{\pgfpoint{0.500000\du}{0\du}}{\pgfpoint{0\du}{0.500000\du}}
\pgfusepath{stroke}
% setfont left to latex
\definecolor{dialinecolor}{rgb}{0.000000, 0.000000, 0.000000}
\pgfsetstrokecolor{dialinecolor}
\node[anchor=west] at (20.500000\du,13.500000\du){};
% setfont left to latex
\definecolor{dialinecolor}{rgb}{0.000000, 0.000000, 0.000000}
\pgfsetstrokecolor{dialinecolor}
\node at (21.000000\du,13.722500\du){M};
\definecolor{dialinecolor}{rgb}{1.000000, 1.000000, 1.000000}
\pgfsetfillcolor{dialinecolor}
\pgfpathellipse{\pgfpoint{23.500000\du}{10.000000\du}}{\pgfpoint{0.500000\du}{0\du}}{\pgfpoint{0\du}{0.500000\du}}
\pgfusepath{fill}
\pgfsetlinewidth{0.100000\du}
\pgfsetdash{}{0pt}
\pgfsetdash{}{0pt}
\definecolor{dialinecolor}{rgb}{0.000000, 0.000000, 0.000000}
\pgfsetstrokecolor{dialinecolor}
\pgfpathellipse{\pgfpoint{23.500000\du}{10.000000\du}}{\pgfpoint{0.500000\du}{0\du}}{\pgfpoint{0\du}{0.500000\du}}
\pgfusepath{stroke}
% setfont left to latex
\definecolor{dialinecolor}{rgb}{0.000000, 0.000000, 0.000000}
\pgfsetstrokecolor{dialinecolor}
\node at (23.500000\du,10.222500\du){B2};
\definecolor{dialinecolor}{rgb}{1.000000, 1.000000, 1.000000}
\pgfsetfillcolor{dialinecolor}
\pgfpathellipse{\pgfpoint{21.000000\du}{9.500000\du}}{\pgfpoint{0.500000\du}{0\du}}{\pgfpoint{0\du}{0.500000\du}}
\pgfusepath{fill}
\pgfsetlinewidth{0.100000\du}
\pgfsetdash{}{0pt}
\pgfsetdash{}{0pt}
\definecolor{dialinecolor}{rgb}{0.000000, 0.000000, 0.000000}
\pgfsetstrokecolor{dialinecolor}
\pgfpathellipse{\pgfpoint{21.000000\du}{9.500000\du}}{\pgfpoint{0.500000\du}{0\du}}{\pgfpoint{0\du}{0.500000\du}}
\pgfusepath{stroke}
% setfont left to latex
\definecolor{dialinecolor}{rgb}{0.000000, 0.000000, 0.000000}
\pgfsetstrokecolor{dialinecolor}
\node at (21.000000\du,9.722500\du){M};
\definecolor{dialinecolor}{rgb}{1.000000, 1.000000, 1.000000}
\pgfsetfillcolor{dialinecolor}
\pgfpathellipse{\pgfpoint{23.000000\du}{6.000000\du}}{\pgfpoint{0.500000\du}{0\du}}{\pgfpoint{0\du}{0.500000\du}}
\pgfusepath{fill}
\pgfsetlinewidth{0.100000\du}
\pgfsetdash{}{0pt}
\pgfsetdash{}{0pt}
\definecolor{dialinecolor}{rgb}{0.000000, 0.000000, 0.000000}
\pgfsetstrokecolor{dialinecolor}
\pgfpathellipse{\pgfpoint{23.000000\du}{6.000000\du}}{\pgfpoint{0.500000\du}{0\du}}{\pgfpoint{0\du}{0.500000\du}}
\pgfusepath{stroke}
% setfont left to latex
\definecolor{dialinecolor}{rgb}{0.000000, 0.000000, 0.000000}
\pgfsetstrokecolor{dialinecolor}
\node at (23.000000\du,6.222500\du){B3};
\definecolor{dialinecolor}{rgb}{1.000000, 1.000000, 1.000000}
\pgfsetfillcolor{dialinecolor}
\pgfpathellipse{\pgfpoint{22.000000\du}{3.500000\du}}{\pgfpoint{0.500000\du}{0\du}}{\pgfpoint{0\du}{0.500000\du}}
\pgfusepath{fill}
\pgfsetlinewidth{0.100000\du}
\pgfsetdash{}{0pt}
\pgfsetdash{}{0pt}
\definecolor{dialinecolor}{rgb}{0.000000, 0.000000, 0.000000}
\pgfsetstrokecolor{dialinecolor}
\pgfpathellipse{\pgfpoint{22.000000\du}{3.500000\du}}{\pgfpoint{0.500000\du}{0\du}}{\pgfpoint{0\du}{0.500000\du}}
\pgfusepath{stroke}
% setfont left to latex
\definecolor{dialinecolor}{rgb}{0.000000, 0.000000, 0.000000}
\pgfsetstrokecolor{dialinecolor}
\node at (22.000000\du,3.722500\du){B3};
\definecolor{dialinecolor}{rgb}{1.000000, 1.000000, 1.000000}
\pgfsetfillcolor{dialinecolor}
\pgfpathellipse{\pgfpoint{25.500000\du}{4.000000\du}}{\pgfpoint{0.500000\du}{0\du}}{\pgfpoint{0\du}{0.500000\du}}
\pgfusepath{fill}
\pgfsetlinewidth{0.100000\du}
\pgfsetdash{}{0pt}
\pgfsetdash{}{0pt}
\definecolor{dialinecolor}{rgb}{0.000000, 0.000000, 0.000000}
\pgfsetstrokecolor{dialinecolor}
\pgfpathellipse{\pgfpoint{25.500000\du}{4.000000\du}}{\pgfpoint{0.500000\du}{0\du}}{\pgfpoint{0\du}{0.500000\du}}
\pgfusepath{stroke}
% setfont left to latex
\definecolor{dialinecolor}{rgb}{0.000000, 0.000000, 0.000000}
\pgfsetstrokecolor{dialinecolor}
\node at (25.500000\du,4.222500\du){B4};
\definecolor{dialinecolor}{rgb}{1.000000, 1.000000, 1.000000}
\pgfsetfillcolor{dialinecolor}
\pgfpathellipse{\pgfpoint{20.500000\du}{1.000000\du}}{\pgfpoint{0.500000\du}{0\du}}{\pgfpoint{0\du}{0.500000\du}}
\pgfusepath{fill}
\pgfsetlinewidth{0.100000\du}
\pgfsetdash{}{0pt}
\pgfsetdash{}{0pt}
\definecolor{dialinecolor}{rgb}{0.000000, 0.000000, 0.000000}
\pgfsetstrokecolor{dialinecolor}
\pgfpathellipse{\pgfpoint{20.500000\du}{1.000000\du}}{\pgfpoint{0.500000\du}{0\du}}{\pgfpoint{0\du}{0.500000\du}}
\pgfusepath{stroke}
% setfont left to latex
\definecolor{dialinecolor}{rgb}{0.000000, 0.000000, 0.000000}
\pgfsetstrokecolor{dialinecolor}
\node at (20.500000\du,1.222500\du){M};
\definecolor{dialinecolor}{rgb}{1.000000, 1.000000, 1.000000}
\pgfsetfillcolor{dialinecolor}
\pgfpathellipse{\pgfpoint{16.500000\du}{1.500000\du}}{\pgfpoint{0.500000\du}{0\du}}{\pgfpoint{0\du}{0.500000\du}}
\pgfusepath{fill}
\pgfsetlinewidth{0.100000\du}
\pgfsetdash{}{0pt}
\pgfsetdash{}{0pt}
\definecolor{dialinecolor}{rgb}{0.000000, 0.000000, 0.000000}
\pgfsetstrokecolor{dialinecolor}
\pgfpathellipse{\pgfpoint{16.500000\du}{1.500000\du}}{\pgfpoint{0.500000\du}{0\du}}{\pgfpoint{0\du}{0.500000\du}}
\pgfusepath{stroke}
% setfont left to latex
\definecolor{dialinecolor}{rgb}{0.000000, 0.000000, 0.000000}
\pgfsetstrokecolor{dialinecolor}
\node at (16.500000\du,1.713188\du){B5};
\definecolor{dialinecolor}{rgb}{1.000000, 1.000000, 1.000000}
\pgfsetfillcolor{dialinecolor}
\pgfpathellipse{\pgfpoint{18.500000\du}{3.500000\du}}{\pgfpoint{0.500000\du}{0\du}}{\pgfpoint{0\du}{0.500000\du}}
\pgfusepath{fill}
\pgfsetlinewidth{0.100000\du}
\pgfsetdash{}{0pt}
\pgfsetdash{}{0pt}
\definecolor{dialinecolor}{rgb}{0.000000, 0.000000, 0.000000}
\pgfsetstrokecolor{dialinecolor}
\pgfpathellipse{\pgfpoint{18.500000\du}{3.500000\du}}{\pgfpoint{0.500000\du}{0\du}}{\pgfpoint{0\du}{0.500000\du}}
\pgfusepath{stroke}
% setfont left to latex
\definecolor{dialinecolor}{rgb}{0.000000, 0.000000, 0.000000}
\pgfsetstrokecolor{dialinecolor}
\node at (18.500000\du,3.713188\du){B5};
\definecolor{dialinecolor}{rgb}{1.000000, 1.000000, 1.000000}
\pgfsetfillcolor{dialinecolor}
\pgfpathellipse{\pgfpoint{20.500000\du}{5.500000\du}}{\pgfpoint{0.500000\du}{0\du}}{\pgfpoint{0\du}{0.500000\du}}
\pgfusepath{fill}
\pgfsetlinewidth{0.100000\du}
\pgfsetdash{}{0pt}
\pgfsetdash{}{0pt}
\definecolor{dialinecolor}{rgb}{0.000000, 0.000000, 0.000000}
\pgfsetstrokecolor{dialinecolor}
\pgfpathellipse{\pgfpoint{20.500000\du}{5.500000\du}}{\pgfpoint{0.500000\du}{0\du}}{\pgfpoint{0\du}{0.500000\du}}
\pgfusepath{stroke}
% setfont left to latex
\definecolor{dialinecolor}{rgb}{0.000000, 0.000000, 0.000000}
\pgfsetstrokecolor{dialinecolor}
\node at (20.500000\du,5.722500\du){M};
\end{tikzpicture}

  \end{figure} 
\end{frame}

\begin{frame}
  \frametitle{Prace se stromem}
  \begin{itemize}
   \item Co je radek?
   \item Co je soubor?
   \item Co je slozka?
   \item Co je commit?
   \item Co je index?
   \item Co je branch?
   \item Co je tag?
   \item Co je reference?
  \end{itemize}
\end{frame}

\subsection{Add / RM}
\begin{frame}
 \frametitle{Add / RM}
  \begin{block}{Add}
   \begin{itemize}
    \item Podepise zmeny a prida je k commitu (soft)
    \item Lzou pridat i soubory z ignore s parametrem --force (nastaveni atd.)
   \end{itemize}
  \end{block}
   
  \begin{block}{RM} 
   \begin{itemize}
    \item Odebere soubory z repozitare
    \item Lze odebrat mekce s parametrem --cached
   \end{itemize}
  \end{block}
\end{frame}
 
\subsection{Commit / Reset}
\begin{frame}
 \frametitle{Commit / Reset}
  \begin{block}{Commit}
   Podepise zmeny, prida zpravu a informace o uzivateli
  \end{block}

  \begin{block}{Reset}
    Odebere soubor/zmenu ze zmen k commitnutí, případně ruší celé commity
    \begin{itemize}
      \item Soft
      \item Hard
      \item HEAD\textasciitilde{}N
      \item Rychly nastroj pro squash(ovani) commitu
    \end{itemize}
  \end{block}
\end{frame}

\subsection{Fast-forward}
\begin{frame}
 \frametitle{Fast-forward}
 \begin{itemize}
  \item Nahraje po sobe jdouci zmeny (push/pull)
  \item Pouziva se hlavne tam, kde se automaticky provadi zmeny
  \item Zjednodusuje prehlednost zmen (mene commitu)
 \end{itemize}
\end{frame}


\subsection{Push / Pull}
\begin{frame}
 \frametitle{Push / Pull}
  Stáhne/nahraje změny z vzdáleného repozitáře
  
  \begin{itemize}
   \item Metadata
   \item Stream objektu
   \item Lisi se prakticky jen, jake se volaji hooky
  \end{itemize}
\end{frame}

\subsection{Rebase vs Merge}
\begin{frame}
  \frametitle{Rebase vs Merge} 
  \begin{itemize}
    \item Lisi se v poradi aplikovani mergu
    \item Stejne co se tyce vysledneho stavu zdrojovych kodu
    \item Vyrazne jine pro code review
    \item Velky rozdil pri automatickych/schvalovanych push/pull/merge requestech
  \end{itemize}
\end{frame}

\subsection{Stash}
\begin{frame}
 \frametitle{Stash}
 \begin{itemize}
  \item Pred pull / push jeste nez se chce udelat opravdovy commit
  \item Pokud se potrebuje aktualizovat, jsou rozpracovane zmeny ktere se nebudou commitovat (gitlab)
 \end{itemize}
\end{frame}

\subsection{Adresovani}
\begin{frame}
 \frametitle{Adresovani}
 \begin{itemize}
  \item refs/heads/master
  \item refs/tags/V0.0.1
  \item refs/remotes/origin/master
 \end{itemize}
\end{frame}


\section{Kooperace}
\begin{frame}
 \frametitle{Zakladni domluva - diktat}
 \begin{itemize}
  \item Jeden tomu veli, protoze vice lidi se uz nemusi shodnout :)
  \item Kde se spravujou ukoly, reference z commitu, closes atd.
  \item Problem s case insenzitive soubory - aneb prokliname windows
  \item Kdo muze mergovat do masteru - merge/push/pull requesty, kazdy sam, atd.
  \item Politika vytvareni/mazani branchu
  \item Fork a k cemu je dobry
  \item Co na to code-review
 \end{itemize}
\end{frame}

\subsection{Mere/pull/push requesty}
\begin{frame}
 \frametitle{Mere/pull/push requesty}
 \begin{itemize}
  \item Jedna se prakticky o tutez vec, jen je ji mozno vyresit N moznostma.
  \item FastForward vs Merge
  \item Co nejmensi pocet commitu, idealne squash(nute) do jednoho komentare - generovani changelogu
  \item Co nejmene primergovani, idealne jen rebasovat master
  \item Automatika nemuze zafungovat vzdy - rebase, merge, push --force
 \end{itemize}
\end{frame}

\subsection{Drobneni projektu}
\begin{frame}
 \frametitle{Drobneni projektu}
 \begin{itemize}
  \item Pouzivat kompozery, instalatory, stahovatory atd. (npm, bower, composer, pip, galaxy atd.)
  \item Pozor na git submodule
  \item Jednotlive modulu jde pouzivat na vice mistech a netreba kopirovat kod
 \end{itemize}
\end{frame}

\section{CI}
\begin{frame}
 \frametitle{CI}
 \begin{itemize}
  \item GitLab-Ci - test
  \item Jenkins
  \item TeamCity
  \item mnoho dalsich
 \end{itemize}
\end{frame}

\subsection{Gitlab}
\begin{frame}
 \frametitle{GitLab}
 https://docs.gitlab.com/ce/ci/yaml/README.html
\begin{block}{.gitlab-ci.yml}
~~stages:\\
~~~~-~test\\
~~job1:\\
~~~~stage:~test\\
~~~~script:\\
~~~~~~- python~test.py\\
~~~~~~- python~-m~unittest
\end{block}
\vspace{0.5cm}

 Testovaci projekt na: http://gitlab.mmik.cz/test/python
 
 \begin{itemize}
  \item Variables
  \item Environments
  \item Runners
 \end{itemize}
\end{frame}



\section{Zaver}
\begin{frame}
 \frametitle{Zaver}
 Gitlab neni jediny a neni to jedina uzitecna sluzba:
 
 \begin{itemize}
  \item Gitosis
  \item GitHub
  \item Jira stash
 \end{itemize} 
 Stejne ale vetsina veci jde resit i jednoduse pres zakladni git server 
\end{frame}

\end{document}
